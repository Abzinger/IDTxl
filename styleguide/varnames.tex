 
%% Basierend auf einer TeXnicCenter-Vorlage von Mark Müller
%%%%%%%%%%%%%%%%%%%%%%%%%%%%%%%%%%%%%%%%%%%%%%%%%%%%%%%%%%%%%%%%%%%%%%%

% Wählen Sie die Optionen aus, indem Sie % vor der Option entfernen  
% Dokumentation des KOMA-Script-Packets: scrguide

%%%%%%%%%%%%%%%%%%%%%%%%%%%%%%%%%%%%%%%%%%%%%%%%%%%%%%%%%%%%%%%%%%%%%%%
%% Optionen zum Layout des Artikels                                  %%
%%%%%%%%%%%%%%%%%%%%%%%%%%%%%%%%%%%%%%%%%%%%%%%%%%%%%%%%%%%%%%%%%%%%%%%
\documentclass[%
%a5paper,							% alle weiteren Papierformat einstellbar
%landscape,						% Querformat
10pt,								% Schriftgröße (12pt, 11pt (Standard))
%BCOR1cm,							% Bindekorrektur, bspw. 1 cm
%DIVcalc,							% führt die Satzspiegelberechnung neu aus
%											  s. scrguide 2.4
%twoside,							% Doppelseiten
%twocolumn,						% zweispaltiger Satz
%halfparskip*,				% Absatzformatierung s. scrguide 3.1
%headsepline,					% Trennline zum Seitenkopf	
%footsepline,					% Trennline zum Seitenfuß
%titlepage,						% Titelei auf eigener Seite
%normalheadings,			% Überschriften etwas kleiner (smallheadings)
%idxtotoc,						% Index im Inhaltsverzeichnis
%liststotoc,					% Abb.- und Tab.verzeichnis im Inhalt
%bibtotoc,						% Literaturverzeichnis im Inhalt
%abstracton,					% Überschrift über der Zusammenfassung an	
%leqno,   						% Nummerierung von Gleichungen links
%fleqn,								% Ausgabe von Gleichungen linksbündig
%draft								% überlangen Zeilen in Ausgabe gekennzeichnet
]
{scrartcl}

%\pagestyle{empty}		% keine Kopf und Fußzeile (k. Seitenzahl)
%\pagestyle{headings}	% lebender Kolumnentitel  


%% Deutsche Anpassungen %%%%%%%%%%%%%%%%%%%%%%%%%%%%%%%%%%%%%
\usepackage[english]{babel}
\usepackage[T1]{fontenc}
\usepackage[ansinew]{inputenc}
\usepackage{listings}
\usepackage{lmodern} %Type1-Schriftart für nicht-englische Texte
\usepackage{color}
\usepackage{url}
\usepackage{float}
%% Packages für Grafiken & Abbildungen %%%%%%%%%%%%%%%%%%%%%%
\usepackage{graphicx} %%Zum Laden von Grafiken
\usepackage{tabularx}
\usepackage{longtable}
\usepackage{textcomp}
\usepackage{hyperref}
%% Bibliographiestil %%%%%%%%%%%%%%%%%%%%%%%%%%%%%%%%%%%%%%%%%%%%%%%%%%
%\usepackage{natbib}

\begin{document}

\pagestyle{empty} %%Keine Kopf-/Fusszeilen auf den ersten Seiten.


%%%%%%%%%%%%%%%%%%%%%%%%%%%%%%%%%%%%%%%%%%%%%%%%%%%%%%%%%%%%%%%%%%%%%%%
%% Ihr Artikel                                                       %%
%%%%%%%%%%%%%%%%%%%%%%%%%%%%%%%%%%%%%%%%%%%%%%%%%%%%%%%%%%%%%%%%%%%%%%%

%% eigene Titelseitengestaltung %%%%%%%%%%%%%%%%%%%%%%%%%%%%%%%%%%%%%%%    
%\begin{titlepage}
%Einsetzen der TXC Vorlage "Deckblatt" möglich
%\end{titlepage}

%% Angaben zur Standardformatierung des Titels %%%%%%%%%%%%%%%%%%%%%%%%
%\titlehead{Titelkopf }
%\subject{Typisierung}
\title{TRENTOOL XL -- Styleguide}
\author{Patricia Wollstadt}
%\and{Der Name des Co-Autoren}
%\thanks{Fußnote}			% entspr. \footnote im Fließtext
%\date{}							% falls anderes, als das aktuelle gewünscht
%\publishers{Herausgeber}

%% Widmungsseite %%%%%%%%%%%%%%%%%%%%%%%%%%%%%%%%%%%%%%%%%%%%%%%%%%%%%%
%\dedication{Widmung}

%\maketitle 						% Titelei wird erzeugt

%% Zusammenfassung nach Titel, vor Inhaltsverzeichnis %%%%%%%%%%%%%%%%%
%\begin{abstract}
% Für eine kurze Zusammenfassung des folgenden Artikels.
% Für die Überschrift s. \documentclass[abstracton].
%\end{abstract}

%% Erzeugung von Verzeichnissen %%%%%%%%%%%%%%%%%%%%%%%%%%%%%%%%%%%%%%%
%\tableofcontents			% Inhaltsverzeichnis
%\listoftables				% Tabellenverzeichnis
%\listoffigures				% Abbildungsverzeichnis


%% Der Text %%%%%%%%%%%%%%%%%%%%%%%%%%%%%%%%%%%%%%%%%%%%%%%%%%%%%%%%%%%

\section{Python Convenstions}

\begin{enumerate}
    \item PEP 8
    \item PEP 257 for docstrings
    \item Google-Styleguide for Python
    \item Auto-Documentation with Sphinx (uses markdown for layout) 
\end{enumerate}

\section{IDTxl Units}

\begin{itemize}
 \item SI units for user input/output
 \item internally, everything is handled in samples, i.e., user input should be translated into samples asap
 \item in general, arrays denote any non-scalar variable
\end{itemize}


\section{IDTxl Variable and Function Names}

\begin{longtable}{p{0.1em}lp{9cm}}
 \multicolumn{3}{l}{\textbf{Measures}} \\ \hline
 & ent & entropy \\
 & cent & conditional entropy \\
 & mi & mutual information \\
 & cmi & conditional mutual information \\
 & multi & multiinformation \\
 & lais & local active information storage \\
 & ais & active information storage \\
 & lte & local transfer entropy \\
 & te & transfer entropy \\
 & mi\_syn & synergistic information\\
 & mi\_unq & unique information\\
 & mi\_shd & shared information\\ 
 &&\\
 \multicolumn{3}{l}{\textbf{Variable Types}} \\ \hline
 & variable & random variable\\
 & process & most generic name for a series of realisation (e.g. realisations over time, i.e., a time series) \\
 & sample & individual realisation of a variable, i.e., one entry in a data array \\
 & source(\_set) & ordered series of realisations of some variable (e.g. a time series) that is investigated as source of information for a second, \textit{target} variable (e.g. in TE estimation); may be a set of vectors \\
 & target & ordered series of realisations of some variable (e.g. a time series) that is investigated as target variable in TE estimation or other directed measures \\
 & current\_value & present sample to be predicted by past states/values (e.g. in AIS or TE estimation) \\ 
 & past/history & past values with respect to the current value/present of a time series (e.g. in TE or AIS estimation for time series) \\
 & conditional & variable to be conditioned on (e.g. in CMI estimation) \\
 &&\\
 \multicolumn{3}{l}{\textbf{Function Names}} \\ \hline 
 & \textlangle measure\textrangle\_calculator\_\textlangle type\textrangle & estimator that takes raw variables or embedded variables as input \\
 &&\\
 \multicolumn{3}{l}{\textbf{Estimator Types}} \\ \hline 
 & kraskov & Kraskov estimator \\
 & kl & Kozachenko-Leonenko estimator \\
 & gaussian & Gaussian estimator \\
 & kernel & Gaussian estimator \\
 &&\\
 \multicolumn{3}{l}{\textbf{Data Properties}} \\ \hline 
 & continuous & variable that takes continuous values \\
 & discrete & variable that takes discrete values \\
 & alphabet\_size & number of symbols in a variable \\
 & replication & repetition of a recording, e.g., trial in neuro-experiment\\
 &&\\
 \multicolumn{3}{l}{\textbf{Algorithm}} \\ \hline
 & idx\_\textlangle variable \textrangle(\_\textlangle set\textrangle) & index of a single sample or set of indices\\
 & candidate(\_set) & potential sample or set of samples for (non-)uniform embedding \\
 & selected\_vars\_* & candidate variables currently included in the conditioning set, * may be either 'full', 'sources', or 'target' to indicate all variables and sub-sets of variables coming from source or target processes respectively \\
 & max\_lag & maximum lag for samples entering the candidate set \\
 & min\_lag & minimum lag for samples entering the candidate set \\
 & embedding\_dimension & dimension of the embedding, i.e., n.o. samples taken from a varibales past to reconstruct a variable's current state \\
 & embedding\_delay & delay of the embedding, i.e., the step size in n.o. samples between two past samples to reconstruct a variable's current state \\
 & theiler\_k & n.o. samples to be excluded in neighbour searches, Theiler correction \\
 & kraskov\_k & n.o. nearest neighbours for the Kraskov estimator \\
 & lag\_max & max. past sample used as a candidate \\
 & lag\_min & min. past sample used as a candidate \\
 & source\_target\_delay & input delay in samples, i.e., something that is passed  to an estimator \\
 & interaction\_delay & output of the estimation (e.g. reconstructed delay in TE estimation) \\
 & surrogate\_\textlangle variable\textrangle & surrogate data set for a variable \\
\end{longtable}



\end{document}
